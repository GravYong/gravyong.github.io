\let\nofiles\relax % This is because res says to not emit aux files,
                   % but lastpage needs aux files.
\documentclass[margin,line]{res}
\usepackage[colorlinks=true]{hyperref}
\usepackage[utf8]{inputenc}
\usepackage[T1]{fontenc}
\usepackage{microtype}
\usepackage{amsmath,amssymb}
\usepackage{lastpage}
\usepackage{fancyhdr}
\usepackage{etaremune}
\usepackage[normalem]{ulem}

\oddsidemargin -.5in
\evensidemargin -.5in
\voffset -25pt
%\topmargin -.2in
\headsep 25pt
\textwidth=6.0in
\textheight=9.0in
\itemsep=0in
\parsep=0in

% Headings
\pagestyle{fancy}
\lhead{Yong Gao --- Curriculum Vitae}
\chead{}
\rhead{\thepage\ of \pageref*{LastPage}}
\lfoot{}
\cfoot{}
\rfoot{}
\renewcommand{\headrulewidth}{0.4pt}
%\renewcommand{\footrulewidth}{0.4pt}

\begin{document}

\newcommand{\myname}{Yong Gao --- Curriculum Vitae}
\newlength{\mynamewidth}
\settowidth{\mynamewidth}{\namefont\myname}

\name{\hspace*{0.5\textwidth}\hspace{-0.5\mynamewidth} \large{\myname}\vspace*{.1in}}
% On the first page, have no header.
\thispagestyle{empty}

\begin{resume}

\section{\sc Contact Information}
%\vspace{.05in}
0.27, Albert Einstein Institute      \hfill email: \href{mailto:gaoyong.physics@pku.edu.cn}{yong.gao@aei.mpg.de}\\
Am Mühlenberg 1, Golm \hfill personal webpage: \href{https://gravyong.github.io/}{gravyong.github.io}\\
Potsdam 14469, Germany                                \hfill telephone: \href{tel:8613811809693}{(86)13811809693}

%%%%%%%%%%%%%%%%%%%%%%%%%%%%%%%%%%%%%%%%%%%%

% Local Variables:
% mode: latex
% TeX-master: "LeoCStein.tex"
% End:
% \hfill personal website: \href{https://gravyong.github.io}{gravyong.github.io}

% \section{\sc Citizenship}
% United States

\section{\sc Education}

{\bf Postdoctoral Researcher,} Max Planck Institute for Gravitational Physics (Albert Einstein Institute), Potsdam, Germany \hfill {\bf September 2023-Present}\\


{\bf Ph.D. candidate, Physics,} Peking University, Beijing, China \hfill {\bf August 2018-Present}\\
\vspace*{-.1in}
\begin{itemize}
\item[ ] Thesis Advisor: Prof.\ Lijing Shao
\item[ ] Thesis Title: {\it Probing Structures of Neutron Stars with Gravitational Waves}
\end{itemize}

{\bf B.S., Physics,} Dalian University of Technology, Dalian, Liaoning Province, China \hfill {\bf July 2018}\\
\vspace*{-.1in}
\begin{itemize}
\item[ ] Degree conferred with honor.
\item[ ] Senior Dissertation Advisors: Prof.\ Renxin Xu and Prof.\ Chong Li
\item[] Dissertation Title: {\it The Electron Distributions of Strangelets in the Thomas-Fermi Model}
% \item[ ] G.P.A.: 3.8/4.0
\end{itemize}


\section{\sc Research Interests}

{\bf Understanding composition and state of matter inside neutron stars (NSs).} Modelling gravitational waves (GWs) from NSs: tidal/spin effects in binary NS and NS-black hole systems, global non-radial oscillations of NSs, mountains on NSs. Studying dynamics and observational consequences of free/forced precession of NSs. Numerical simulations of compact binary mergers involving NSs.

{\bf Testing strong-field gravity.} Modelling GW waveform from compact binaries and oscillating compact objects beyond general relativity. Constructing timing model and testing gravity with pulsar timing. Studying the structures of rotating, tidally-deformed, and oscillating NSs in alternative theories of gravity. Numerical simulations of compact binaries in alternative theories of gravity.



\section{\sc Honors and Awards}
{\bf Principal Scholarship,} Peking University \hfill {\bf 2022-2023}\\
\\
{\bf Tung Scholarship,} Peking University \hfill {\bf 2021--2022}\\
\\
{\bf Merit Student,} Peking University \hfill {\bf 2021--2022}\\
\\
{\bf The Second Prize for Oral Presentation,} Physics Five Universities \hfill {\bf April 2021}\\
\\
{\bf Vela Prize for Oral Presentation,} FAST/Future Pulsar Symposium 9 \hfill {\bf August 2020}\\
\\
{\bf National Scholarship,} Peking University \hfill {\bf 2019--2020}\\
% \\
% {\bf Merit Student,} Peking University \hfill {\bf 2019--2020}\\
\\
{\bf Excellent Teaching Assistant Award,} Peking University \hfill {\bf 2019--2020}\\
\\
{\bf Principal Scholarship,} Peking University \hfill {\bf 2018--2019}\\
\\
{\bf Learning Excellence Award (First Prize),} Dalian University of Technology \hfill {\bf 2015--2016}\\
\\
% {\bf National Encouragement Scholarship,} Dalian University of Technology \hfill {\bf 2015--2016}\\
% \\
% {\bf National Encouragement Scholarship,} Dalian University of Technology \hfill {\bf 2014--2015}\\
% \\
% Added to improve page breaks

\vspace{-1em}

\section{\sc Teaching Experience}

{\bf Lecturer}, Max Planck Institute for Gravitational Physics 
\vspace*{.05in}
\begin{itemize}
  \item[ ] \href{https://springschool.aei.mpg.de/}{Introductions to the Fundamentals of Neutron Stars} \hfill {\bf Spring 2025}
\end{itemize}


{\bf Teaching Assistant}, Peking University
\vspace*{.05in}
\begin{itemize}
  \item[ ] \href{http://friendshao.github.io/teaching/ced22/}{Electrodynamics (B)} \hfill {\bf Fall 2022}
\item[ ] \href{http://friendshao.github.io/teaching/phy21/}{General Physics I}, $^{*}${\small incl. Mechanics $\&$ Electromagnetism} \hfill {\bf Fall 2021}
\item[ ] \href{http://friendshao.github.io/teaching/thmech19/}{Theoretical Mechanics (A)}, {\small{\bf Excellent Teaching Assistant Award}} \hfill {\bf Fall 2019}
\end{itemize}

\section{\sc Co-advised Students}

{\bf Ph.D. Student}, Peking University
\vspace*{.05in}
\begin{itemize}
\item[] {\bf Hongbo Li}, co-advised with Prof. Lijing Shao and Prof. Renxin Xu 
\hfill {\bf 2021--present} \\
\vspace*{.05in}
{\it  Oscillations of neutron stars and gravitational-wave asteroseismology}
\end{itemize}
{\bf Undergraduate Students}, Peking University
\vspace*{.05in}
\begin{itemize}
\item[] {\bf Haoyang Qi}, co-advised with Prof. Lijing Shao
\hfill {\bf 2021-Present}\\
{\it Constraints on ultralight dark matter with pulsar timing}
\item[] {\bf Huimei Wang}, co-advised with Prof. Lijing Shao 
\hfill {\bf 2020-2021}\\
{\it Undergraduate Dissertation: The structure of neutron stars with anisotropic pressure}
\item[] {\bf Jingyuan Deng}, co-advised with Prof. Lijing Shao 
\hfill {\bf 2020-2021}\\
{\it Undergraduate Dissertation: Forced precession of neutron stars}
\item[] {\bf Zexin Hu}, co-advised with Prof. Lijing Shao
\hfill {\bf 2020-2021}\\
{\it Scalarized neutron stars in massive scalar-tensor gravity}
\end{itemize}

\section{\sc Computer Skills}
%{\bf Languages---}%
Proficient in {\sc Mathematica}, Julia, Fortran, Python, and Matlab. Experience in C, Bash, and HPC.\\
Markup languages: \LaTeX, Markdown.

% {\bf Operating systems---}%
% Mac OS, Linux/*nix.

{\bf Code development---} Most contributions can be found at \url{https://github.com/GravYong}.

\section{\sc Professional Activities, Outreach, and Service}
{\bf KAGRA Collaboration}
\vspace*{.05in}
% \begin{itemize}
%   \item[] {Member of Compact Binaries Coalescence (CBC) Group} \hfill{\bf 2018--Present}
% \end{itemize}
\begin{itemize}
\item[] Member of KAGRA Future Strategy Committee (FSC) \hfill{\bf 2021--2023}
\end{itemize}

% {\bf Conference organizer}
% \vspace*{.05in}

{\bf Chair of conference session/group meeting}
\vspace*{.05in}
\begin{itemize}
\item[]
\href{https://gwwiki.icrr.u-tokyo.ac.jp/JGWwiki/KAGRA/KSC/FSC/FWG/1stMeeting}{KAGRA Future Working Group 1st Open Meeting} ({\it online}) \hfill {\bf November 2021}
\end{itemize}
\begin{itemize}
  \item[]
  {Chair of the }\href{https://kiaagravity.github.io}{\sc Kiaagravity group meeting}   \hfill {\bf 2020-2021}
  \end{itemize}

{\bf Journal referee}
\vspace*{.05in}
% \hspace*{1em}
\begin{itemize}
  \item[]
  Classical and Quantum Gravity (CQG)
  \hfill {\bf 2021-Present}
  \item[]
  Research in Astronomy and Astrophysics (RAA)
  \hfill {\bf 2021-Present}
  \item[]
  Science China Physics, Mechanics $\&$ Astronomy (SCPMA)
  \hfill {\bf 2021-Present}
\end{itemize}

% {\bf Participated Grants}
% \vspace*{.05in}\\
% \hspace*{1em}
% ???
% Twiddle this for spacing
% \newpage

\ifx\nopubs\undefined
\newcommand{\arxiv}[1]{[\href{http://arxiv.org/abs/#1}{arXiv:#1}]}
\newcommand{\citeCount}[1]{(#1 citations)}
\renewcommand{\citeCount}[1]{}

\newcounter{numPubs}
\newcounter{pubCounter}

\setcounter{numPubs}{26}
\setcounter{pubCounter}{\value{numPubs}}

\section{\sc Submitted Publications}
% \addtocounter{pubCounter}{-1}
\begin{etaremune}[start=\value{pubCounter}]
  \item
  M.~Z.~Han, {\bf Y.~Gao}, K.~Kiuchi and M.~Shibata,
  {\it Dependence of post-merger properties on the thermal heating efficiency in neutron star mergers},
  submitted to Phys. Rev. D
  \arxiv{2504.08514}
  \item
  {\bf Y.~Gao}, K.~Hayashi, K.~Kiuchi, A.~T.~L.~Lam, H.~J.~Kuan and M.~Shibata,
  {\it Convective stability analysis of massive neutron stars formed in binary mergers},
  submitted to Phys. Rev. D
  \arxiv{2501.19053}
\end{etaremune}

% \section{\sc Accepted Publications}
% \addtocounter{pubCounter}{0}
% \begin{etaremune}[start=\value{pubCounter}]
  % \item  
  % {\bf Y. Gao},
  % L. Shao,
  % G. Desvignes,
  % D. I. Jones,
  % M. Kramer,
  % G. Yim,
  % {\it Precession of magnetars: dynamical evolutions and modulations on polarized electromagnetic waves},
  % accepted by MNRAS
  % \arxiv{2211.17087}.
  % \item 
  % {\bf Y. Gao}, 
  % R. Xu, 
  % L. Shao, 
  % {\it Precession of spheroids under Lorentz violation and observational consequences for neutron stars}, 
  % in Proceedings of the Ninth Meeting on CPT and Lorentz Symmetry, in press.
%   \setcounter{pubCounter}{\value{enumi}}
% \end{etaremune}


\section{\sc Refereed Publications}
\begin{etaremune}
  \item 
  A.~T.~L.~Lam, {\bf Y.~Gao}, H.~J.~Kuan, M.~Shibata, K.~Van Aelst and K.~Kiuchi,
  {\it Accessing universal relations of binary neutron star waveforms in massive scalar-tensor theory},
  Phys. Rev. Lett. {\bf 134}, 15, 151402
  \arxiv{2410.00137}
  \item 
  Z.~Wang, {\bf Y.~Gao}, D.~Liang, J.~Zhao and L.~Shao,
  {\it Vetting quark-star models with gravitational waves in the hierarchical Bayesian framework},
  JCAP {\bf 11}, 038
  \arxiv{2409.11103}
  \item 
  Y. Liu, H. B. Li, {\bf Y. Gao}, L. Shao, Z. Hu, Effects from dark matter halos on X-ray pulsar pulse profiles, Phys. Rev. D {\bf 110}, 083018 \arxiv{2408.04425}
  \item 
  S. C. Chen, {\bf Y. Gao}, E. P. Zhou, R.-X. Xu, {\it Free energy of anisotropic strangeon stars}, Res. Astron. Astrophys. {\bf 24}, 025005 \arxiv{2305.19687}.
  \item 
  E. P. Zhou, 
  {\bf Y. Gao}, 
  Y. R. Zhou, 
  X. Y. Lai, L. Shao, W. Y. Wang, S.-L. Xiong, R.-X. Xu, S. X. Yi, H. Yue, Z. Zhang, 
  {\it The precursor of GRB211211A: a tide-induced giant quake?}, Res. Astron. Astrophys. {\bf 24}, 025019
  \arxiv{2305.19687}
  \item 
  G.~Yim, {\bf Y.~Gao}, Y.~Kang, L.~Shao and R.~Xu,
  {\it Continuous gravitational waves from trapped magnetar ejecta and the connection to glitches and antiglitches},
  Mon. Not. Roy. Astron. Soc. {\bf 527}, 2, 2379-2392 
  \arxiv{2308.01588}
  \item 
  C.~Zhang, 
  {\bf Y.~Gao}, 
  C.~J.~Xia,
  R.~Xu,
  {\it Rescaling strange-cluster stars and its implications on gravitational-wave echoes},
  Phys. Rev. D {\bf 108}, 6, 063002
  \arxiv{2305.13323}
  \item 
  Y.~Kang, C.~Liu, J.~P.~Zhu, {\bf Y.~Gao}, L.~Shao, B.~Zhang, H.~Sun, Y.~H.~I.~Yin and B.~B.~Zhang,
  {\it Prospects for detecting neutron star\textendash{}white dwarf mergers with decihertz gravitational-wave observatories},
  Mon. Not. Roy. Astron. Soc. {\bf 528}, 3, 5309-5322
  \arxiv{2309.16991}
  \item 
  {\bf Y. Gao},
  L. Shao,
  Jan Steinhoff,
  {\it A tight universal relation between the shape eccentricity and the moment of inertia for rotating neutron stars},
  Astrophys. J. {\bf 954}, {1}, 16
  \arxiv{2303.14130}
  \item
  H.-B. Li, 
  {\bf Y. Gao}, 
  L. Shao, 
  R.-X. Xu, 
  {\it The g-mode of neutron stars in Pseudo-Newtonian gravity}, 
  Phys. Rev. D 108 {\bf 6}, 064005
  \arxiv{2302.03856}.
  \item
  G. Desvignes, 
  P. Weltevrede, 
  {\bf Y. Gao}, 
  D. I. Jones, 
  M. Kramer, 
  M. Caleb, 
  R. Karuppusamy, 
  L. Levin, 
  K. Liu, 
  A. G. Lyne, 
  L. Shao, 
  B. Stappers, 
  {\it A freely precessing magnetar following an X-ray outburst}, 
  Nature Astron. {\bf 8}, 617-627.
  \item  
  {\bf Y. Gao},
  L. Shao,
  G. Desvignes,
  D. I. Jones,
  M. Kramer,
  G. Yim,
  {\it Precession of magnetars: dynamical evolutions and modulations on polarized electromagnetic waves},
   Mon. Not. Roy. Astron. Soc. {\bf 1}, 1080-1097 
  \arxiv{2211.17087}.
  \item 
  {\bf Y. Gao}, 
  R. Xu, 
  L. Shao, 
  {\it Precession of spheroids under Lorentz violation and observational consequences for neutron stars}, 
  in Proceedings of the Ninth Meeting on CPT and Lorentz Symmetry, published.

\item
  {\bf Y. Gao}, 
  X.-Y. Lai, 
  L. Shao, 
  R.-X. Xu,
  (2022)
  {\it Rotation and deformation of strangeon stars in the Lennard-Jones model}, 
  \href{https://doi.org/10.1093/mnras/stab3181}{Mon. Not. R. Astron. Soc. {\bf 509},~2758}
  \arxiv{2109.13234}.
\item 
  {\bf Y. Gao}, 
  L. Shao, 
  R. Xu, 
  L. Sun, 
  C. Liu, 
  R.-X. Xu,
  (2020) 
  {\it Triaxially-deformed freely-precessing neutron stars: continuous electromagnetic and gravitational radiation}, 
  \href{https://doi.org/10.1093/mnras/staa2476}{Mon. Not. R. Astron. Soc. {\bf 498},~1826}
  \arxiv{2007.02528}.
\item 
  {\bf Y. Gao}, 
  L. Shao,
  (2021) 
  {\it Precession of triaxially deformed neutron stars}, 
  \href{https://doi.org/10.1002/asna.202113935}{Astron. Nachr. {\bf 342},~364}
  \arxiv{2011.04472}.
\item 
  Z. Hu, 
  {\bf Y. Gao },
  R. Xu, 
  L. Shao, 
  (2021)
  {\it Scalarized neutron stars in massive scalar-tensor gravity: X-ray pulsars and tidal deformability}, 
  \href{https://doi.org/10.1103/PhysRevD.104.104014}{Phys. Rev. D {\bf 104},~104014}
  \arxiv{2109.13453}.
\item
  H.-B. Li, 
  {\bf Y. Gao },
  L. Shao, 
  R.-X. Xu, 
  R. Xu, 
  (2022)
  {\it Oscillation modes and gravitational waves from strangeon stars},
  \href{https://academic.oup.com/mnras/advance-article-abstract/doi/10.1093/mnras/stac2622/6705433}{Mon. Not. R. Astron. Soc. {\bf 516},~6172}
  \arxiv{2206.09407}.
\item 
  R. Xu, 
  {\bf Y. Gao}, 
  L. Shao, 
  (2022)
  {\it Neutron stars in massive scalar-Gauss-Bonnet gravity: Spherical structure and time-independent perturbations}, 
  \href{https://doi.org/10.1103/PhysRevD.105.024003}{Phys. Rev. D {\bf 105},~024003}
  \arxiv{2111.06561}.
  \item 
  R. Xu, 
  {\bf Y. Gao}, 
  L. Shao, 
  (2021)
  Signature of Lorentz violation in continuous gravitational-wave spectra of ellipsoidal neutron stars, 
  \href{https://doi.org/10.3390/galaxies9010012}{Galaxies {\bf 9},~12}
  \arxiv{2101.09431}.  
  \setcounter{pubCounter}{\value{enumi}}

\item 
  R. Xu, 
  {\bf Y. Gao}, 
  L. Shao, 
  (2021)
  {\it Precession of spheroids under Lorentz violation and observational consequences for neutron stars}, 
  \href{https://doi.org/10.1103/PhysRevD.103.084028}{Phys. Rev. D {\bf 103},~084028}
  \arxiv{2012.01320}.
\item 
  R. Xu, 
  {\bf Y. Gao}, 
  L. Shao, 
  (2020)
  {\it Strong-field effects in massive scalar-tensor gravity for slowly spinning neutron stars and application to X-ray pulsar pulse profiles}, 
  \href{https://doi.org/10.1103/PhysRevD.102.064057}{Phys. Rev. D {\bf 102}, 064057}
  \arxiv{2007.10080}.
\item 
  J. Zhao, 
  L. Shao, 
  {\bf Y. Gao}, 
  C. Liu, 
  Z. Cao, 
  B.-Q. Ma,
  (2021) 
  {\it Probing dipole radiation from binary neutron stars with ground-based laser-interferometer and atom-interferometer gravitational-wave observatories}, 
  \href{https://doi.org/10.1103/PhysRevD.104.084008}{Phys. Rev. D {\bf 104},~084008}
  \arxiv{2106.04883}.
\item 
  C. Liu, 
  L. Shao, 
  J. Zhao, 
  {\bf Y. Gao}, 
  (2020)
  {\it Multiband observation of LIGO/Virgo binary black hole mergers in the gravitational-wave transient catalog GWTC-1}, 
  \href{https://doi.org/10.1093/mnras/staa1512}{Mon. Not. R. Astron. Soc. {\bf 496},~182}
  \arxiv{2004.12096}.
\end{etaremune}




\else
%
\fi

\section{\sc Popular Science Articles}

\begin{etaremune}
  \item {\bf Y. Gao}, L. Shao, R.-X. Xu, (2019) \href{https://gravyong.github.io/files/BNS_Popular.pdf}{The waltz of a binary neutron star system} ({an article about GW170817, \it in Chinese}).
  \item {\bf Y. Gao}, (2022) \href{https://gravyong.github.io/files/NS_Structure_Popular.pdf}{The structures of neutron stars} ({an article about dense matter in neutron stars, \it in Chinese}).
  \item {\bf Y. Gao}, L. Shao, (2022) \href{https://www.ligo.org/science/Publication-O3bTGR/translations/science-summary-chinese-simplified.pdf}{Does Einstein's theory of gravity hold up to the latest \\
  LIGO/VIRGO/KAGRA observations?} ({\bf translated} from \href{https://www.ligo.org/science/Publication-O3bTGR/}{the English version}).
\end{etaremune}


%\newcommand{\playsymbol}{\framebox[1.3\width]{$\blacktriangleright$}}
\newcommand{\playsymbol}{$\blacktriangleright$}
\section{\sc Invited Talks}
\begin{etaremune}
  \item 
  Yangzhou University, School of Physics Science and Technology, Seminar
  \hfill{} 
  September 2022
\item 
  Peking University, School of Physics, CuiYing Graduate Student Salon
  \hfill{} 
  February 2021
\item
  Max Planck Institut f$\rm \ddot{u}$r Gravitationsphysik Colloquium ({\it online})
  \hfill{}
  September 2020
\item
  University of Tartu, Theoretical Physics Laboratory Colloquium ({\it online})
  \hfill{}
  October 2020
\end{etaremune}

\section{\sc Contributed Talks}
\begin{etaremune}
\item
  SKA Pulsar Science Symposium 2022
  \hfill{}
  August 2022
\item  
  FAST/Future Pulsar Symposium 11
  \hfill 
  August 2022
\item  
  Summer Science Day, KIAA, Peking University
  \hfill 
  July 2022
\item 
  The 60th Anniversary of X-Ray Astronomy ({\it online})
  \hfill 
  June 2022
\item 
  Ninth Meeting on CPT and Lorentz Symmetry ({\it online})
  \hfill
  May 2022
\item  
  FAST/Future Pulsar Symposium 10
  \hfill 
  July 2021
\item  
  Gravitation and Relativistic Astrophysics, Chinese Physical Society
  \hfill 
  April 2021
\item 
  Gravitation and Cosmology Symposium
  \hfill 
  December 2020
\item 
  FAST/Future Pulsar Symposium 9
  \hfill 
  August 2020
\end{etaremune}

% \section{\sc Participated Meetings (no talks)}
% \begin{etaremune}
% \item 
% To be determined
% \end{etaremune}
%%%%%%%%%%%%%%%%%%%%%%%%%%%%%%%%%%%%%%%%%%%%

% Local Variables:
% mode: latex
% TeX-master: "LeoCStein.tex"
% End:


\newpage{}
% \vspace{0.1in}

\section{\sc References}
\vspace*{.05in}
\parbox{\textwidth}{%
{\bf Lijing~Shao,} Assistant Professor of Kavli Institute for Astronomy and Astrophysics, Peking University \\
K217, Kavli Institute for Astronomy and Astrophysics  \\
5 Yiheyuan Road, Haidian District \\
Beijing 100871, P. R. China \\
email: \href{mailto:lshao@pku.edu.cn}{lshao@pku.edu.cn} \\
office phone: \href{tel: 86-10-6275-8461}{ 86-10-6275-8461}}
\par
\parbox{\textwidth}{%
{\bf Renxin~Xu,} Professor of Physics, Peking University\\
2912, Science Teaching Building No.~2, Department of Astronomy\\
5 Yiheyuan Road, Haidian District \\
Beijing 100871, P. R. China \\
email: \href{mailto:r.x.xu@pku.edu.cn}{r.x.xu@pku.edu.cn} \\
office phone: \href{tel:86-10-6275-8631}{86-10-6275-8631}}
\par
\parbox{\textwidth}{%
{\bf David Ian Jones,} Professor of Mathematical Physics, University of Southampton\\
B54, West Highfield Campus\\
University Road, SO17 1BJ\\
Southampton, United Kingdom \\
email: \href{mailto:d.i.jones@soton.ac.uk}{d.i.jones@soton.ac.uk} \\
office phone: \href{tel:44-23-8059-4829}{44-23-8059-4829}}
\par
\parbox{\textwidth}{%
{\bf Gregory Desvignes,} Postdoctoral Researcher of Radio Astronomy, Max Planck Institute for Radio Astronomy\\
Auf dem Hügel 69\\
D-53121 Bonn, Germany\\
email: \href{mailto:gdesvignes.astro@gmail.com}{gdesvignes.astro@gmail.com}}
% office phone: \href{tel:44-23-8059-4829}{44-23-8059-4829}}

\end{resume}
\end{document}

% Local Variables:
% mode: latex
% End: