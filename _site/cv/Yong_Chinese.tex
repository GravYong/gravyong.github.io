\documentclass[margin,line]{res_mod}
\let\nofiles\relax
\usepackage{ctex}
\usepackage[scheme=plain,fontset=fandol]{ctex}
\usepackage[colorlinks=true]{hyperref}
% \usepackage{fontspec}
\usepackage{microtype}
\usepackage{amsmath,amssymb}
\usepackage{lastpage}
\usepackage{fancyhdr}
\usepackage{etaremune}
\usepackage[normalem]{ulem}

% 页面格式
\oddsidemargin -.5in
\evensidemargin -.5in
\voffset -25pt
\headsep 25pt
\textwidth=6.0in
\textheight=9.0in
\itemsep=0in
\parsep=0in

% 页眉页脚
\pagestyle{fancy}
\lhead{高勇 —— 个人简历}
\chead{}
\rhead{第\thepage 页,共\pageref*{LastPage}页}
\lfoot{}
\cfoot{}
\rfoot{}
\renewcommand{\headrulewidth}{0.4pt}

\begin{document}


\newcommand{\myname}{高勇 --- 个人简历}
\newlength{\mynamewidth}
\settowidth{\mynamewidth}{\namefont\myname}
\name{\hspace*{0.5\textwidth}\hspace{-0.5\mynamewidth} \large{\myname}\vspace*{.1in}}
% On the first page, have no header.
\thispagestyle{empty}

\begin{resume}

\section{联系方式}
0.27, 阿尔伯特·爱因斯坦研究所      \hfill 邮箱: \href{mailto:gaoyong.physics@pku.edu.cn}{yong.gao@aei.mpg.de}\\
Am Mühlenberg 1, Golm \hfill 个人主页: \href{https://gravyong.github.io/}{gravyong.github.io}\\
波茨坦 14476, 德国                                \hfill 电话: \href{tel:8613811809693}{(86)13811809693}

\section{教育背景}

{\bf 博士后研究员},德国马克斯·普朗克引力物理研究所(阿尔伯特·爱因斯坦研究所),波茨坦,德国 \hfill {\bf 2023年9月 — 至今} 

{\bf 博士,物理学},北京大学,北京,中国 \hfill {\bf 2018年8月 — 2023年8月} \\
\vspace*{-.1in}
\begin{itemize}
  \item[ ] 导师:邵立晶 研究员
  \item[ ] 博士论文题目:\textit{多信使天文学时代下中子星结构和自旋进动的研究}
\end{itemize}

{\bf 学士,物理学},大连理工大学,辽宁大连,中国 \hfill {\bf 2014年9月 — 2018年7月} \\
\vspace*{-.1in}
\begin{itemize}
  \item[ ] 本科论文导师:徐仁新 教授、李崇 教授
  \item[ ] 本科论文题目:\textit{托马斯-费米模型中奇异核的电子分布}
\end{itemize}


\section{研究方向}

\textbf{中子星的内部结构:} 中子星的结构建模、非径向振荡与星震学、中子星形变与连续引力波、中子星自由/强迫进动的动力学与观测表现,以及致密双星并合的数值模拟。

\textbf{强引力场检验:} 替代引力理论中致密天体的结构以及双星并合的数值相对论模拟。

\section{所获荣誉}
\begin{itemize}
  \item 北京大学校长奖学金 \hfill \textbf{2022–2023}
  \item 北京大学董氏奖学金 \hfill \textbf{2021–2022}
  \item 北京大学三好学生 \hfill \textbf{2021–2022}
  \item 五校联盟口头报告二等奖 \hfill \textbf{2021年4月}
  \item FAST脉冲星大会“Vela”优秀口头报告奖 \hfill \textbf{2020年8月}
  \item 北京大学国家奖学金, \hfill \textbf{2019–2020}
  \item 北京大学优秀助教奖 \hfill \textbf{2019–2020}
  \item 北京大学校长奖学金 \hfill \textbf{2018–2019}
  \item 大连理工大学学习优秀一等奖 \hfill \textbf{2015–2016}
\end{itemize}


\section{教学经历}

\textbf{讲师},德国马克斯普朗克引力物理研究所 \hfill \textbf{2025年春季} \\
Jürgen Ehlers 春季学校:\textit{中子星基础简介}

\vspace{0.8em}

\textbf{助教},北京大学
\begin{itemize}
  \item 电动力学 B,获优秀助教奖 \hfill \textbf{2022年秋季}
  \item 普通物理 I(涵盖力学与电磁学) \hfill \textbf{2021年秋季}
  \item 理论力学 A,获优秀助教奖 \hfill \textbf{2019年秋季}
\end{itemize}



\section{联合指导学生}

% \textbf{博士生:} 李洪波(2021–2023),合导师:邵立晶、徐仁新\
% 研究方向:中子星振荡与引力波星震学

\textbf{本科生:}
\begin{itemize}
  \item 齐昊洋,脉冲星计时对超轻暗物质的限制 \hfill \textbf{2021–2022}
  \item 王慧美,各向异性压强条件下的中子星结构建模 \hfill \textbf{2020–2021}
  \item 邓景元,中子星的受迫进动动力学研究 \hfill \textbf{2020–2021}
  \item 胡泽昕,标量-张量引力理论中的中子星结构分析 \hfill \textbf{2020–2021}
\end{itemize}


\section{计算技能}
熟练:\textsc{Mathematica}, Julia, Fortran, Python;熟悉 C、Bash、高性能计算;排版语言:\LaTeX, Markdown。个人代码主页:\url{https://github.com/GravYong}

\section{学术活动与服务}
\begin{itemize}
  \item KAGRA 合作组成员 \hfill \textbf{2021–2023}
  \item 主持会议:KAGRA 未来发展组第一次线上公开会议 \hfill \textbf{2021年11月}
  \item 主持讨论:Kiaagravity 小组会议 \hfill \textbf{2020–2021}
  \item 审稿人:《Classical and Quantum Gravity》、《Research in Astronomy and Astrophysics》、\\
  \phantom{审稿人:}《Science China Physics, Mechanics \& Astronomy》
\end{itemize}


\ifx\nopubs\undefined
\newcommand{\arxiv}[1]{[\href{http://arxiv.org/abs/#1}{arXiv:#1}]}
\newcommand{\citeCount}[1]{(#1 citations)}
\renewcommand{\citeCount}[1]{}

\newcounter{numPubs}
\newcounter{pubCounter}

\setcounter{numPubs}{28}
\setcounter{pubCounter}{\value{numPubs}}

\section{ 投稿论文}
% \addtocounter{pubCounter}{-1}
\begin{etaremune}[start=\value{pubCounter}]
  \item 
  Z.~Hu, {\bf Y.~Gao} and L.~Shao,
  {\it Linear analysis of I-C-Love universal relations for neutron stars},
  submitted to Phys. Rev. D,
  \arxiv{2505.13110}
  \item
  M.~Z.~Han, {\bf Y.~Gao}, K.~Kiuchi and M.~Shibata,
  {\it Dependence of post-merger properties on the thermal heating efficiency in neutron star mergers},
  submitted to Phys. Rev. D
  \arxiv{2504.08514}
  \item
  {\bf Y.~Gao}, K.~Hayashi, K.~Kiuchi, A.~T.~L.~Lam, H.~J.~Kuan and M.~Shibata,
  {\it Convective stability analysis of massive neutron stars formed in binary mergers},
  submitted to Phys. Rev. D
  \arxiv{2501.19053}
\end{etaremune}

% \section{\sc Accepted Publications}
% \addtocounter{pubCounter}{0}
% \begin{etaremune}[start=\value{pubCounter}]
  % \item  
  % {\bf Y. Gao},
  % L. Shao,
  % G. Desvignes,
  % D. I. Jones,
  % M. Kramer,
  % G. Yim,
  % {\it Precession of magnetars: dynamical evolutions and modulations on polarized electromagnetic waves},
  % accepted by MNRAS
  % \arxiv{2211.17087}.
  % \item 
  % {\bf Y. Gao}, 
  % R. Xu, 
  % L. Shao, 
  % {\it Precession of spheroids under Lorentz violation and observational consequences for neutron stars}, 
  % in Proceedings of the Ninth Meeting on CPT and Lorentz Symmetry, in press.
%   \setcounter{pubCounter}{\value{enumi}}
% \end{etaremune}


\section{发表论文}
\begin{etaremune}
  \item 
  A.~T.~L.~Lam, {\bf Y.~Gao}, H.~J.~Kuan, M.~Shibata, K.~Van Aelst and K.~Kiuchi,
  {\it Accessing universal relations of binary neutron star waveforms in massive scalar-tensor theory},
  \href{https://journals.aps.org/prl/abstract/10.1103/PhysRevLett.134.151402}{Phys. Rev. Lett. {\bf 134}, 15, 151402}
  \arxiv{2410.00137}
  \item 
  Z.~Wang, {\bf Y.~Gao}, D.~Liang, J.~Zhao and L.~Shao,
  {\it Vetting quark-star models with gravitational waves in the hierarchical Bayesian framework},
  \href{https://iopscience.iop.org/article/10.1088/1475-7516/2024/11/038}{JCAP {\bf 11}, 038}
  \arxiv{2409.11103}
  \item 
  Y. Liu, H. B. Li, {\bf Y. Gao}, L. Shao, Z. Hu, Effects from dark matter halos on X-ray pulsar pulse profiles, Phys. Rev. D {\bf 110}, 083018 \arxiv{2408.04425}
  \item 
  S. C. Chen, {\bf Y. Gao}, E. P. Zhou, R.-X. Xu, {\it Free energy of anisotropic strangeon stars}, Res. Astron. Astrophys. {\bf 24}, 025005 \arxiv{2305.19687}.
  \item 
  E. P. Zhou, 
  {\bf Y. Gao}, 
  Y. R. Zhou, 
  X. Y. Lai, L. Shao, W. Y. Wang, S.-L. Xiong, R.-X. Xu, S. X. Yi, H. Yue, Z. Zhang, 
  {\it The precursor of GRB211211A: a tide-induced giant quake?}, Res. Astron. Astrophys. {\bf 24}, 025019
  \arxiv{2305.19687}
  \item 
  G.~Yim, {\bf Y.~Gao}, Y.~Kang, L.~Shao and R.~Xu,
  {\it Continuous gravitational waves from trapped magnetar ejecta and the connection to glitches and antiglitches},
  Mon. Not. Roy. Astron. Soc. {\bf 527}, 2, 2379-2392 
  \arxiv{2308.01588}
  \item 
  C.~Zhang, 
  {\bf Y.~Gao}, 
  C.~J.~Xia,
  R.~Xu,
  {\it Rescaling strange-cluster stars and its implications on gravitational-wave echoes},
  Phys. Rev. D {\bf 108}, 6, 063002
  \arxiv{2305.13323}
  \item 
  Y.~Kang, C.~Liu, J.~P.~Zhu, {\bf Y.~Gao}, L.~Shao, B.~Zhang, H.~Sun, Y.~H.~I.~Yin and B.~B.~Zhang,
  {\it Prospects for detecting neutron star\textendash{}white dwarf mergers with decihertz gravitational-wave observatories},
  Mon. Not. Roy. Astron. Soc. {\bf 528}, 3, 5309-5322
  \arxiv{2309.16991}
  \item 
  {\bf Y. Gao},
  L. Shao,
  Jan Steinhoff,
  {\it A tight universal relation between the shape eccentricity and the moment of inertia for rotating neutron stars},
  \href{https://iopscience.iop.org/article/10.3847/1538-4357/ace776}{Astrophys. J. {\bf 954}, {1}, 16}
  \arxiv{2303.14130}
  \item 
  H.~Liu, {\bf Y.~Gao}, Z.~Li, A.~Dohi, W.~Wang, G.~Lv and R.~Xu,
  {\it EOS-dependent millihertz quasi-periodic oscillation in low-mass X-ray binary},
  Mon. Not. Roy. Astron. Soc. {\bf 525}, 2, 2054-2068
  \arxiv{2308.05288}
  \item
  H.-B. Li, 
  {\bf Y. Gao}, 
  L. Shao, 
  R.-X. Xu, 
  {\it The g-mode of neutron stars in Pseudo-Newtonian gravity}, 
  Phys. Rev. D 108 {\bf 6}, 064005
  \arxiv{2302.03856}.
  \item
  G. Desvignes, 
  P. Weltevrede, 
  {\bf Y. Gao}, 
  D. I. Jones, 
  M. Kramer, 
  M. Caleb, 
  R. Karuppusamy, 
  L. Levin, 
  K. Liu, 
  A. G. Lyne, 
  L. Shao, 
  B. Stappers, 
  {\it A freely precessing magnetar following an X-ray outburst}, 
  Nature Astron. {\bf 8}, 617-627.
  \item  
  {\bf Y. Gao},
  L. Shao,
  G. Desvignes,
  D. I. Jones,
  M. Kramer,
  G. Yim,
  {\it Precession of magnetars: dynamical evolutions and modulations on polarized electromagnetic waves}, \href{https://doi.org/10.1093/mnras/stac3546}{Mon. Not. Roy. Astron. Soc. {\bf 1}, 1080-1097}
  \arxiv{2211.17087}.
  \item 
  {\bf Y. Gao}, 
  R. Xu, 
  L. Shao, 
  {\it Precession of spheroids under Lorentz violation and observational consequences for neutron stars}, 
  in Proceedings of the Ninth Meeting on CPT and Lorentz Symmetry, published.

\item
  {\bf Y. Gao}, 
  X.-Y. Lai, 
  L. Shao, 
  R.-X. Xu,
  (2022)
  {\it Rotation and deformation of strangeon stars in the Lennard-Jones model}, 
  \href{https://doi.org/10.1093/mnras/stab3181}{Mon. Not. R. Astron. Soc. {\bf 509},~2758}
  \arxiv{2109.13234}.
\item 
  {\bf Y. Gao}, 
  L. Shao, 
  R. Xu, 
  L. Sun, 
  C. Liu, 
  R.-X. Xu,
  (2020) 
  {\it Triaxially-deformed freely-precessing neutron stars: continuous electromagnetic and gravitational radiation}, 
  \href{https://doi.org/10.1093/mnras/staa2476}{Mon. Not. R. Astron. Soc. {\bf 498},~1826}
  \arxiv{2007.02528}.
\item 
  {\bf Y. Gao}, 
  L. Shao,
  (2021) 
  {\it Precession of triaxially deformed neutron stars}, 
  \href{https://doi.org/10.1002/asna.202113935}{Astron. Nachr. {\bf 342},~364}
  \arxiv{2011.04472}.
\item 
  Z. Hu, 
  {\bf Y. Gao },
  R. Xu, 
  L. Shao, 
  (2021)
  {\it Scalarized neutron stars in massive scalar-tensor gravity: X-ray pulsars and tidal deformability}, 
  \href{https://doi.org/10.1103/PhysRevD.104.104014}{Phys. Rev. D {\bf 104},~104014}
  \arxiv{2109.13453}.
\item
  H.-B. Li, 
  {\bf Y. Gao },
  L. Shao, 
  R.-X. Xu, 
  R. Xu, 
  (2022)
  {\it Oscillation modes and gravitational waves from strangeon stars},
  \href{https://academic.oup.com/mnras/advance-article-abstract/doi/10.1093/mnras/stac2622/6705433}{Mon. Not. R. Astron. Soc. {\bf 516},~6172}
  \arxiv{2206.09407}.
\item 
  R. Xu, 
  {\bf Y. Gao}, 
  L. Shao, 
  (2022)
  {\it Neutron stars in massive scalar-Gauss-Bonnet gravity: Spherical structure and time-independent perturbations}, 
  \href{https://doi.org/10.1103/PhysRevD.105.024003}{Phys. Rev. D {\bf 105},~024003}
  \arxiv{2111.06561}.
  \item 
  R. Xu, 
  {\bf Y. Gao}, 
  L. Shao, 
  (2021)
  Signature of Lorentz violation in continuous gravitational-wave spectra of ellipsoidal neutron stars, 
  \href{https://doi.org/10.3390/galaxies9010012}{Galaxies {\bf 9},~12}
  \arxiv{2101.09431}.  
  \setcounter{pubCounter}{\value{enumi}}

\item 
  R. Xu, 
  {\bf Y. Gao}, 
  L. Shao, 
  (2021)
  {\it Precession of spheroids under Lorentz violation and observational consequences for neutron stars}, 
  \href{https://doi.org/10.1103/PhysRevD.103.084028}{Phys. Rev. D {\bf 103},~084028}
  \arxiv{2012.01320}.
\item 
  R. Xu, 
  {\bf Y. Gao}, 
  L. Shao, 
  (2020)
  {\it Strong-field effects in massive scalar-tensor gravity for slowly spinning neutron stars and application to X-ray pulsar pulse profiles}, 
  \href{https://doi.org/10.1103/PhysRevD.102.064057}{Phys. Rev. D {\bf 102}, 064057}
  \arxiv{2007.10080}.
\item 
  J. Zhao, 
  L. Shao, 
  {\bf Y. Gao}, 
  C. Liu, 
  Z. Cao, 
  B.-Q. Ma,
  (2021) 
  {\it Probing dipole radiation from binary neutron stars with ground-based laser-interferometer and atom-interferometer gravitational-wave observatories}, 
  \href{https://doi.org/10.1103/PhysRevD.104.084008}{Phys. Rev. D {\bf 104},~084008}
  \arxiv{2106.04883}.
\item 
  C. Liu, 
  L. Shao, 
  J. Zhao, 
  {\bf Y. Gao}, 
  (2020)
  {\it Multiband observation of LIGO/Virgo binary black hole mergers in the gravitational-wave transient catalog GWTC-1}, 
  \href{https://doi.org/10.1093/mnras/staa1512}{Mon. Not. R. Astron. Soc. {\bf 496},~182}
  \arxiv{2004.12096}.
\end{etaremune}




\else
%
\fi

\section{科普文章}
\begin{etaremune}
\item 高勇, 邵立晶, 徐仁新(2019):\href{https://gravyong.github.io/files/BNS_Popular.pdf}{《双中子星圆舞曲》(中文科普)}
\item 高勇(2022):\href{https://gravyong.github.io/files/NS_Structure_Popular.pdf}{《中子星内部结构》(中文科普)}
\item 高勇, 邵立晶(2022):\href{https://www.ligo.org/science/Publication-O3bTGR/translations/science-summary-chinese-simplified.pdf}{《爱因斯坦的引力理论依然站得住脚吗?》(LIGO官方翻译)}
\item 高勇,Gregory Desvignes, 邵立晶 (2024):\href{https://pure.mpg.de/rest/items/item_3612895_2/component/file_3612896/content}{一颗自由进动的磁星}(中文科普)
\end{etaremune}

\section{报告与学术演讲}
\vspace*{.4in}
\newcommand{\playsymbol}{$\blacktriangleright$}
\section{\sc 受邀报告}
\begin{etaremune}
  \item 扬州大学 物理科学与技术学院 学术报告 \hfill{} 2024年12月
  \item 扬州大学 物理科学与技术学院 学术报告 \hfill{} 2022年9月
  \item 北京大学 物理学院 翠英研究生沙龙 \hfill{} 2021年2月
  \item 德国马克斯-普朗克引力物理研究所报告({\it 在线}) \hfill{} 2020年9月
  \item 爱沙尼亚塔尔图大学 理论物理实验室系列报告({\it 在线}) \hfill{} 2020年10月
\end{etaremune}

\section{\sc 会议报告}
\begin{etaremune}
  \item 研讨会:利用引力波解码状态方程(波兰 华沙大学) \hfill{} 2024年8月
  \item SKA 脉冲星科学研讨会 2022 \hfill{} 2022年8月
  \item 第十一届FAST/未来脉冲星研讨会\hfill{} 2022年8月
  \item 夏季科学日,北京大学科维理天文与天体物理研究所 \hfill{} 2022年7月
  \item X射线天文学60周年纪念大会({\it 在线}) \hfill{} 2022年6月
  \item 第九届CPT与洛伦兹对称性研讨会({\it 在线}) \hfill{} 2022年5月
  \item 第十届FAST/未来脉冲星研讨会  \hfill{} 2021年7月
  \item 中国物理学会 引力与相对论天体物理分会年会 \hfill{} 2021年4月
  \item 引力与宇宙学研讨会 \hfill{} 2020年12月
  \item 第九届FAST/未来脉冲星研讨会 \hfill{} 2020年8月
\end{etaremune}



\newpage{}
\section{推荐人信息}
\vspace*{.05in}
\parbox{\textwidth}{%
{\bf 邵立晶,} 北京大学科维理天文与天体物理研究所 副教授 \\
北京大学科维理天文与天体物理研究所 K217 室 \\
北京市海淀区颐和园路5号 \\
邮编:100871,中国 \\
电子邮箱:\href{mailto:lshao@pku.edu.cn}{lshao@pku.edu.cn} \\
办公电话:\href{tel: 86-10-6275-8461}{86-10-6275-8461}}
\par
\parbox{\textwidth}{%
{\bf Masaru Shibata,} 马克斯·普朗克引力物理研究所(阿尔伯特·爱因斯坦研究所)所长 \&\\
计算相对论天体物理部 主任 \\
阿尔伯特·爱因斯坦研究所 1.18 室 \\
Am Mühlenberg 1 \\
邮编:14476,德国 波茨坦 \\
电子邮箱:\href{mailto:masaru.shibata@aei.mpg.de}{masaru.shibata@aei.mpg.de} \\
办公电话:\href{tel:49-331-567-7222}{49-331-567-7222}}
\par
\parbox{\textwidth}{%
{\bf 徐仁新,} 北京大学物理学院 天文学系 教授 \\
北京大学理科教学楼二号楼 2912 室 \\
北京市海淀区颐和园路5号 \\
邮编:100871,中国 \\
电子邮箱:\href{mailto:r.x.xu@pku.edu.cn}{r.x.xu@pku.edu.cn} \\
办公电话:\href{tel:86-10-6275-8631}{86-10-6275-8631}}
\par
\parbox{\textwidth}{%
{\bf David Ian Jones,} 南安普顿大学 数理物理 教授 \\
West Highfield 校区 B54 室 \\
University Road, SO17 1BJ \\
英国 南安普顿 \\
电子邮箱:\href{mailto:d.i.jones@soton.ac.uk}{d.i.jones@soton.ac.uk} \\
办公电话:\href{tel:44-23-8059-4829}{44-23-8059-4829}}
\par
\parbox{\textwidth}{%
{\bf Gregory Desvignes,} 马克斯·普朗克射电天文研究所 博士后研究员 \\
Auf dem Hügel 69 \\
邮编:D-53121,德国 波恩 \\
电子邮箱:\href{mailto:gdesvignes.astro@gmail.com}{gdesvignes.astro@gmail.com}}

\end{resume}
\end{document}