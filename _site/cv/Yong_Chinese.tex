\documentclass[margin,line]{res_mod}
\let\nofiles\relax
\usepackage{ctex}
\usepackage[scheme=plain,fontset=fandol]{ctex}
\usepackage[colorlinks=true]{hyperref}
% \usepackage{fontspec}
\usepackage{microtype}
\usepackage{amsmath,amssymb}
\usepackage{lastpage}
\usepackage{fancyhdr}
\usepackage{etaremune}
\usepackage[normalem]{ulem}

% 页面格式
\oddsidemargin -.5in
\evensidemargin -.5in
\voffset -25pt
\headsep 25pt
\textwidth=6.0in
\textheight=9.0in
\itemsep=0in
\parsep=0in

% 页眉页脚
\pagestyle{fancy}
\lhead{高勇 —— 个人简历}
\chead{}
\rhead{第\thepage 页,共\pageref*{LastPage}页}
\lfoot{}
\cfoot{}
\rfoot{}
\renewcommand{\headrulewidth}{0.4pt}

\begin{document}


\newcommand{\myname}{高勇 --- 个人简历}
\newlength{\mynamewidth}
\settowidth{\mynamewidth}{\namefont\myname}
\name{\hspace*{0.5\textwidth}\hspace{-0.5\mynamewidth} \large{\myname}\vspace*{.1in}}
% On the first page, have no header.
\thispagestyle{empty}

\begin{resume}

\section{联系方式}
0.27, 阿尔伯特·爱因斯坦研究所      \hfill 邮箱: \href{mailto:gaoyong.physics@pku.edu.cn}{yong.gao@aei.mpg.de}\\
Am Mühlenberg 1, Golm \hfill 个人主页: \href{https://gravyong.github.io/}{gravyong.github.io}\\
波茨坦 14476, 德国                                \hfill 电话: \href{tel:8613811809693}{(86)13811809693}

\section{教育背景}

{\bf 博士后研究员},德国马克斯·普朗克引力物理研究所(阿尔伯特·爱因斯坦研究所),波茨坦,德国 \hfill {\bf 2023年9月 — 至今} 

{\bf 博士,物理学},北京大学,北京,中国 \hfill {\bf 2018年8月 — 2023年8月} \\
\vspace*{-.1in}
\begin{itemize}
  \item[ ] 导师:邵立晶 研究员
  \item[ ] 博士论文题目:\textit{多信使天文学时代下中子星结构和自旋进动的研究}
\end{itemize}

{\bf 学士,物理学},大连理工大学,辽宁大连,中国 \hfill {\bf 2014年9月 — 2018年7月} \\
\vspace*{-.1in}
\begin{itemize}
  \item[ ] 本科论文导师:徐仁新 教授、李崇 教授
  \item[ ] 本科论文题目:\textit{托马斯-费米模型中奇异核的电子分布}
\end{itemize}


\section{研究方向}

\textbf{中子星的内部结构:} 中子星的结构建模、非径向振荡与星震学、中子星形变与连续引力波、中子星自由/强迫进动的动力学与观测表现,以及致密双星并合的数值模拟。

\textbf{强引力场检验:} 在广义相对论以外的引力理论中开展致密天体结构与引力波形建模研究;构建脉冲星计时模型以进行引力理论检验;探讨旋转中子星的潮汐形变与振荡结构;并开展替代理论下的数值相对论模拟。

\section{所获荣誉}
\begin{itemize}
  \item 北京大学校长奖学金 \hfill \textbf{2022–2023}
  \item 北京大学董氏奖学金 \hfill \textbf{2021–2022}
  \item 北京大学三好学生 \hfill \textbf{2021–2022}
  \item 五校联赛口头报告二等奖 \hfill \textbf{2021年4月}
  \item FAST脉冲星大会“Vela”优秀口头报告奖 \hfill \textbf{2020年8月}
  \item 国家奖学金,北京大学 \hfill \textbf{2019–2020}
  \item 优秀助教奖,北京大学 \hfill \textbf{2019–2020}
  \item 北京大学校长奖学金 \hfill \textbf{2018–2019}
  \item 大连理工大学学习优秀一等奖 \hfill \textbf{2015–2016}
\end{itemize}


\section{教学经历}

\textbf{讲师},德国马克斯普朗克引力物理研究所 \hfill \textbf{2025年春季} \\
Jürgen Ehlers 春季学校:\textit{中子星基础简介}

\vspace{0.8em}

\textbf{助教},北京大学
\begin{itemize}
  \item 电动力学 B,获优秀助教奖 \hfill \textbf{2022年秋季}
  \item 普通物理 I(涵盖力学与电磁学) \hfill \textbf{2021年秋季}
  \item 理论力学 A,获优秀助教奖 \hfill \textbf{2019年秋季}
\end{itemize}



\section{联合指导学生}

% \textbf{博士生:} 李洪波(2021–2023),合导师:邵立晶、徐仁新\
% 研究方向:中子星振荡与引力波星震学

\textbf{本科生:}
\begin{itemize}
  \item 齐昊洋,脉冲星计时对超轻暗物质的限制 \hfill \textbf{2021–2022}
  \item 王慧美,各向异性压强条件下的中子星结构建模 \hfill \textbf{2020–2021}
  \item 邓景元,中子星的受迫进动动力学研究 \hfill \textbf{2020–2021}
  \item 胡泽昕,标量-张量引力理论中的中子星结构分析 \hfill \textbf{2020–2021}
\end{itemize}


\section{计算技能}
熟练:\textsc{Mathematica}, Julia, Fortran, Python;熟悉 C、Bash、高性能计算;排版语言:\LaTeX, Markdown。个人代码主页:\url{https://github.com/GravYong}

\section{学术活动与服务}
\begin{itemize}
  \item KAGRA 合作组成员 \hfill \textbf{2021–2023}
  \item 主持会议:KAGRA 未来发展组第一次线上公开会议 \hfill \textbf{2021年11月}
  \item 主持讨论:Kiaagravity 小组会议 \hfill \textbf{2020–2021}
  \item 审稿人:《Classical and Quantum Gravity》、《Research in Astronomy and Astrophysics》、\\
  \phantom{审稿人:}《Science China Physics, Mechanics \& Astronomy》
\end{itemize}


\ifx\nopubs\undefined
\input{PubsContent_c.tex}
\else
%
\fi

\section{科普文章}
\begin{etaremune}
\item 高勇, 邵立晶, 徐仁新(2019):\href{https://gravyong.github.io/files/BNS_Popular.pdf}{《双中子星圆舞曲》(中文科普)}
\item 高勇(2022):\href{https://gravyong.github.io/files/NS_Structure_Popular.pdf}{《中子星内部结构》(中文科普)}
\item 高勇, 邵立晶(2022):\href{https://www.ligo.org/science/Publication-O3bTGR/translations/science-summary-chinese-simplified.pdf}{《爱因斯坦的引力理论依然站得住脚吗?》(LIGO官方翻译)}
\item 高勇,Gregory Desvignes, 邵立晶 (2024):\href{https://pure.mpg.de/rest/items/item_3612895_2/component/file_3612896/content}{一颗自由进动的磁星}(中文科普)
\end{etaremune}

\section{报告与学术演讲}
\vspace*{.4in}
\newcommand{\playsymbol}{$\blacktriangleright$}
\section{\sc 受邀报告}
\begin{etaremune}
  \item 扬州大学 物理科学与技术学院 学术报告 \hfill{} 2024年12月
  \item 扬州大学 物理科学与技术学院 学术报告 \hfill{} 2022年9月
  \item 北京大学 物理学院 翠英研究生沙龙 \hfill{} 2021年2月
  \item 德国马克斯-普朗克引力物理研究所报告({\it 在线}) \hfill{} 2020年9月
  \item 爱沙尼亚塔尔图大学 理论物理实验室系列报告({\it 在线}) \hfill{} 2020年10月
\end{etaremune}

\section{\sc 会议报告}
\begin{etaremune}
  \item 研讨会:利用天体引力波解码状态方程(波兰 华沙大学) \hfill{} 2024年8月
  \item SKA 脉冲星科学研讨会 2022 \hfill{} 2022年8月
  \item FAST / 未来脉冲星研讨会 第十一届 \hfill{} 2022年8月
  \item 夏季科学日,北京大学科维理天文与天体物理研究所 \hfill{} 2022年7月
  \item X 射线天文学 60 周年纪念大会({\it 在线}) \hfill{} 2022年6月
  \item 第九届 CPT 与洛伦兹对称性研讨会({\it 在线}) \hfill{} 2022年5月
  \item FAST / 未来脉冲星研讨会 第十届 \hfill{} 2021年7月
  \item 中国物理学会 引力与相对论天体物理分会年会 \hfill{} 2021年4月
  \item 引力与宇宙学研讨会 \hfill{} 2020年12月
  \item FAST / 未来脉冲星研讨会 第九届 \hfill{} 2020年8月
\end{etaremune}



\newpage{}
\section{推荐人信息}
\vspace*{.05in}
\parbox{\textwidth}{%
{\bf 邵立晶,} 北京大学科维理天文与天体物理研究所 副教授 \\
北京大学科维理天文与天体物理研究所 K217 室 \\
北京市海淀区颐和园路5号 \\
邮编:100871,中国 \\
电子邮箱:\href{mailto:lshao@pku.edu.cn}{lshao@pku.edu.cn} \\
办公电话:\href{tel: 86-10-6275-8461}{86-10-6275-8461}}
\par
\parbox{\textwidth}{%
{\bf Masaru Shibata,} 马克斯·普朗克引力物理研究所(阿尔伯特·爱因斯坦研究所)所长 \&\\
计算相对论天体物理部 主任 \\
阿尔伯特·爱因斯坦研究所 1.18 室 \\
Am Mühlenberg 1 \\
邮编:14476,德国 波茨坦 \\
电子邮箱:\href{mailto:masaru.shibata@aei.mpg.de}{masaru.shibata@aei.mpg.de} \\
办公电话:\href{tel:49-331-567-7222}{49-331-567-7222}}
\par
\parbox{\textwidth}{%
{\bf 徐仁新,} 北京大学物理学院 天文学系 教授 \\
北京大学理科教学楼二号楼 2912 室 \\
北京市海淀区颐和园路5号 \\
邮编:100871,中国 \\
电子邮箱:\href{mailto:r.x.xu@pku.edu.cn}{r.x.xu@pku.edu.cn} \\
办公电话:\href{tel:86-10-6275-8631}{86-10-6275-8631}}
\par
\parbox{\textwidth}{%
{\bf David Ian Jones,} 南安普顿大学 数理物理 教授 \\
West Highfield 校区 B54 室 \\
University Road, SO17 1BJ \\
英国 南安普顿 \\
电子邮箱:\href{mailto:d.i.jones@soton.ac.uk}{d.i.jones@soton.ac.uk} \\
办公电话:\href{tel:44-23-8059-4829}{44-23-8059-4829}}
\par
\parbox{\textwidth}{%
{\bf Gregory Desvignes,} 马克斯·普朗克射电天文研究所 博士后研究员 \\
Auf dem Hügel 69 \\
邮编:D-53121,德国 波恩 \\
电子邮箱:\href{mailto:gdesvignes.astro@gmail.com}{gdesvignes.astro@gmail.com}}

\end{resume}
\end{document}